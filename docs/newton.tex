\documentclass[aps,pra,12pt,showpacs,showkeys,nofootinbib,superscriptaddress,longbibliography]{revtex4}

\usepackage[utf8]{inputenc}
\usepackage{amssymb,amsmath,amsfonts,amsthm}
\usepackage{verbatim}
\usepackage{enumerate}
\usepackage{tensor}
\usepackage{graphicx}
%\usepackage{hyperref}

\renewcommand{\baselinestretch}{1}  % cancels any possible double spacing


% bra and ket: 
\newcommand{\bra}[1]{\mbox{$\langle #1|$}}
\newcommand{\ket}[1]{\mbox{$|#1\rangle$}}
\newcommand{\braket}[2]{\mbox{$\langle #1|#2\rangle$}}
\newcommand{\ketbra}[2]{\mbox{$|#1\rangle\langle #2|$}}

\newcommand{\inprod}[2]{\mbox{$\langle #1, #2\rangle$}}

\newcommand{\comm}[2]{\ensuremath{\left[#1, \: #2\right]}}
\newcommand{\acomm}[2]{\ensuremath{\left\{#1, \: #2\right\}}}

\newcommand{\gate}[1]{\ensuremath{\text{\sc #1}}}

\DeclareMathOperator{\Tr}{Tr}
\DeclareMathOperator{\Real}{Re}
\DeclareMathOperator{\Imag}{Im}
\DeclareMathOperator{\Span}{span}
\DeclareMathOperator{\diag}{diag}
\DeclareMathOperator{\Hom}{Hom} % set of homomorphisms between two spaces/groups
\DeclareMathOperator{\Aut}{Aut} % automorphism group
\DeclareMathOperator{\End}{End} % set of endomorphisms

\DeclareMathOperator{\VEC}{vec}

\DeclareMathOperator{\id}{id} % identity map
\newcommand{\I}{\openone}     % identity operator
\newcommand{\R}{{\mathbb R}}  % real numbers
\newcommand{\C}{{\mathbb C}}  % complex numbers
\newcommand{\hilb}[1]{\ensuremath{\mathcal{#1}}} % Hilbert space
\newcommand{\swap}{{\sf{SWAP}}}
\newcommand{\ie}{i.e.}
\newcommand{\mf}[1]{\mathfrak{#1}}
\newcommand{\wrt}[1]{\: \mathrm{d}#1}  % integral with respect to

\newcommand{\pde}[2]{\frac{\partial #1}{\partial #2}}

\newcommand{\be}{\begin{equation}}
\newcommand{\ee}{\end{equation}}

\newcommand{\Figref}[1]{Fig.~\ref{#1}}


\begin{document}
\title{Notes on the Newton-Raphson quantum gate optimization method}

\author{Ville Bergholm}
\email{ville.bergholm@iki.fi}
\affiliation{Department of Chemistry, Technische Universität München, Germany}

% \pacs{}
% \keywords{tensor network, invariant} not needed as they are embedded into the PDF.  

\begin{abstract}
Detailed derivation of the key results required by the Newton-Raphson
quantum gate optimization algorithm.
\end{abstract}
\maketitle

\section{Introduction}

\subsection{Definitions}
Assume a system with the Hamiltonian
\be
i H(t) = A +\sum_{r=1}^R B_r f_r(t),
\ee
where $A$~is the drift term, $B_r$~are the control terms and
$f_r(t)$ are the control fields. They can be discretized by expanding
them in a basis~$\{b_k(t)\}_k$:
\be
f_r(t) = \sum_{k=1}^K \alpha_{rk} b_k(t).
\ee
The simplest choice is the piecewise constant basis:
\be
b_k(t) = 
\begin{cases}
  1 & \text{when} \quad t_{k-1} \le t < t_k,\\
  0 & \text{otherwise}.
\end{cases}
\ee
In this case the propagator is of a simple product form:
\begin{align*}
U_f(T)
&= \mathcal{T} \exp\left(\frac{1}{i\hbar} \int_0^T H(t) \wrt{t} \right)
=  \mathcal{T} \exp\left(- \frac{1}{\hbar} \sum_{n=1}^N \int_{t_{n-1}}^{t_n} \left(A +\sum_{rk} B_r \alpha_{rk} b_k(t)\right) \wrt{t} \right)\\
&= \mathcal{T} \exp\left(- \frac{1}{\hbar} \sum_{n=1}^N \int_{t_{n-1}}^{t_n} \left(A +\sum_{r} B_r \alpha_{rn} \right) \wrt{t} \right)
= \mathcal{T} \exp\left(\frac{1}{i\hbar} \sum_{n=1}^N \tau_n H_n \right)\\
&= \exp \left(\frac{\tau_N H_N}{i\hbar}\right)
\exp \left(\frac{\tau_{N-1} H_{N-1}}{i\hbar}\right) \cdots
\exp \left(\frac{\tau_1 H_1}{i\hbar}\right)
= U_N U_{N-1} \cdots U_1.
\end{align*}


\subsection{$\exp$ and $\log$}
$\mf{u}(N)$ is the real vector space of $N \times N$ anti-Hermitian
matrices, and also a Lie algebra. We use the Hilbert-Schmidt inner
product
\be
\inprod{A}{B} := \Tr(A^\dagger B),
\ee
and the induced Frobenius norm.

\begin{align*}
\exp &: \mf{u}(N) \to U(N),\\
\log &: U(N) \to \mf{u}(N).
\end{align*}
Any $A \in \mf{u}(N)$ or $U \in U(N)$ is normal, and thus we may use
the eigendecomposition:
\begin{align*}
A &= \sum_k i \alpha_k \ketbra{\lambda_k}{\lambda_k}, \quad \alpha_k \in \R,\\
\exp(A) &= \sum_k \exp(i \alpha_k) \ketbra{\lambda_k}{\lambda_k}.
\end{align*}
$\log$ is defined to use the branch which minimizes $\|\log(A)\|$, which
results in the eigenvalues of~$\log(U)$ to reside in~$i (\pi, \pi]$.



\subsection{Jacobian}
Given a function~$f: \R^a \to \R^b$, its Jacobian is defined as
\be
J(f): \R^a \to \Hom(\R^a, \R^b).
\ee
$f$ is approximated by the
linear function
$f(x+y) \approx f(x) + J(f)|_x \: y$
near~$x \in \R^a$.
The Jacobian is given by the matrix
\be
J_{jk}(f) := \pde{f_j}{x_k}.
\ee
Given a function~$g: \R^b \to \R^c$, we can use the chain rule to
obtain the Jacobian for the function $g \circ f$:
\be
J(g \circ f)|_{x} = J(g)|_{f(x)} J(f)|_{x}
\ee
(matrix product of two Jacobian matrices).

Now let's add linear transformations before and after~$f$:
\begin{align*}
f' &:= B \circ f \circ A,\\
J(f')|_x &= J(B)|_{f(A(x))} J(f)|_{A(x)} J(A)|_{x} = B J(f)|_{A(x)} A,
\end{align*}
since the Jacobian of a linear map at any point is just the corresponding constant matrix.
Hence, if $A$ and $B$ are basis changes (invertible), $J(f')|_x$ is
invertible whenever $J(f)|_{A(x)}$ is.


\subsection{Exact gradient of the~$\exp$ function}


Assume $H(\vec{u}) = A +\sum_{r} u_{r} B_r$ is normal for all choices of~$\vec{u}$:
\be
H(\vec{u}) = \sum_k \lambda_k \ketbra{\lambda_k}{\lambda_k},
\ee
where everything depends implicitly on~$\vec{u}$.
NOTE! Actually we don't use this property at all below, it is enough
to assume that $H$ has eigenvectors and -values.
If $H$ is not normal there won't be an orthonormal basis of eigenvectors, which
hurts us later though.


\begin{align*}
\bra{\lambda_a} \pde{\exp(H(\vec{u}))}{u_{r}} \ket{\lambda_b} 
&=
\bra{\lambda_a} \left. \left(\pde{}{x} \exp(H(\vec{u}) +x
B_r)\right)\right|_{x = 0}  \ket{\lambda_b}\\
&=
\left. \bra{\lambda_a} \left(\pde{}{x} e^{H +x B_r}\right)
\ket{\lambda_b} \right|_{x = 0}\\
&=
\left. \bra{\lambda_a} \left(\pde{}{x} \sum_{n=0}^\infty
\frac{1}{n!}(H +x B_r)^n \right) \ket{\lambda_b} \right|_{x = 0}\\
&=
\left. \bra{\lambda_a} \left(\sum_{n=1}^\infty \frac{1}{n!}
\sum_{q=1}^n (H +x B_r)^{q-1} B_r (H +x B_r)^{n-q} \right) \ket{\lambda_b} \right|_{x = 0}\\
&=
\bra{\lambda_a} \left(\sum_{n=1}^\infty \frac{1}{n!}
\sum_{q=1}^n H^{q-1} B_r H^{n-q} \right) \ket{\lambda_b}\\
%&=
%\left(\sum_{n=1}^\infty \frac{1}{n!}
%\sum_{q=1}^n \lambda_a^{q-1} \bra{\lambda_a} B_r \ket{\lambda_b} \lambda_b^{n-q} \right)\\
&=
\bra{\lambda_a} B_r \ket{\lambda_b} \sum_{n=1}^\infty \frac{1}{n!}
\sum_{q=1}^n \lambda_a^{q-1} \lambda_b^{n-q}.
\end{align*}
If $\lambda_a = \lambda_b$, this gives~$\bra{\lambda_a} B_r \ket{\lambda_b} e^{\lambda_b}$.
Otherwise one of them is nonzero (assuming $\lambda_b \neq 0$ here)
and we have
\begin{align}
\notag
\bra{\lambda_a} \pde{\exp(H(\vec{u}))}{u_{r}} \ket{\lambda_b} 
&=
\bra{\lambda_a} B_r \ket{\lambda_b} \sum_{n=1}^\infty \frac{1}{n!}
\lambda_b^{n-1} \sum_{q=1}^n \left(\frac{\lambda_a}{\lambda_b} \right)^{q-1}
=
\bra{\lambda_a} B_r \ket{\lambda_b} \sum_{n=1}^\infty \frac{1}{n!}
\lambda_b^{n-1} \frac{(\lambda_a/\lambda_b)^n-1}{\lambda_a/\lambda_b - 1}\\
\notag
&=
\bra{\lambda_a} B_r \ket{\lambda_b} \sum_{n=1}^\infty \frac{1}{n!}
\frac{\lambda_a^n - \lambda_b^n}{\lambda_a - \lambda_b}
=
\bra{\lambda_a} B_r \ket{\lambda_b}
\frac{e^{\lambda_a} - e^{\lambda_b}}{\lambda_a - \lambda_b}\\
\notag
&=
\bra{\lambda_a} B_r \ket{\lambda_b} e^{\lambda_b} \gamma(\lambda_a - \lambda_b)
=
\bra{\lambda_a} B_r \ket{\lambda_b} e^{\lambda_a} \gamma(\lambda_b - \lambda_a)\\
&=
\bra{\lambda_a} B_r \ket{\lambda_b} e^{\lambda_a} \Gamma_{ab}
= \bra{\lambda_a} B_r \ket{\lambda_b} \zeta_{ab}
\label{eq:expgrad}
\end{align}
where
\be
\zeta_{ab} := \frac{e^{\lambda_a} - e^{\lambda_b}}{\lambda_a - \lambda_b},
\qquad
\Gamma_{ab} := \gamma(\lambda_b-\lambda_a) \qquad \text{and} \qquad
\gamma(z) := \frac{e^z - 1}{z}.
\ee
By continuously extending the defintion to~$\gamma(0) := 1$, this
expression also works in the case where~$\lambda_a = \lambda_b$.



\section{The linearization function $\mf{L}$}

Let $V$ be the target gate, $U_f$ the actual propagator and $P$ a
projector into the traceless subspace of~$\mf{u}(N)$. The gate
optimization problem is now reduced to finding the root(s) of the function
\be
\mf{L}(f) := P \log(V^\dagger U_f).
\ee




\subsection{Jacobian}


Define
\be
q: U(N) \to U(N), \quad q := \exp \circ \log = \id_{U(N)}.
\ee
Using the chain rule we obtain
\be
J(q)|_{U} = J(\exp)|_{\log(U)} J(\log)|_{U} = \I
\ee
% \I: which vector space? depends on the parameterizations

If $J(\exp)|_{\log(U)}$ is invertible we obtain
\be
\label{eq:invJ}
J(\log)|_{U} =  (J(\exp)|_{\log(U)})^{-1}.
\ee

Now, assume we express the elements of
$\mf{u}(N)$
in an antihermitian basis~$\{B_j\}_{j=1}^{N^2}$ using real coefficients:
\be
A = \sum_j a_j B_j = \sum_k \lambda_k \ketbra{\lambda_k}{\lambda_k}
\ee
(where the latter expression is the eigendecomposition of a
particular~$A \in \mf{u}(N)$).
Expanding the operator $\exp(A)$ in the eigenbasis~$\{\ket{\lambda_k}\}_{k=1}^N$ and then vectorizing it, we
obtain the Jacobian matrix
\begin{align}
J_{ab,j}(\exp)|_A
:= \left.\left(\bra{\lambda_a}  \pde{\exp(A)}{a_j} \ket{\lambda_b} \right) \right|_A
= e^{\lambda_a} \Gamma_{ab} \bra{\lambda_a} B_j \ket{\lambda_b},
\end{align}
($a,b$: eigenbasis, $j$: antihermitian basis)
where the final expression was obtained using Eq.~\eqref{eq:expgrad} from the previous section.
Now define $\Lambda := \sum_i \ketbra{\lambda_i}{i}$.
\begin{align}
\notag
\VEC^{-1}\left(J(\exp)|_A \: \VEC(D)\right)
&= \sum_{abj} \ketbra{\lambda_a}{\lambda_b} \: J_{ab,j}(\exp)|_A \: \VEC(B_j)^\dagger \VEC(D)\\
\notag
&= \sum_{abj} \ketbra{\lambda_a}{\lambda_b}
e^{\lambda_a} \Gamma_{ab} \bra{\lambda_a} B_j \ket{\lambda_b} \: \inprod{B_j}{D}\\
\notag
&= \sum_{ab} \ketbra{\lambda_a}{\lambda_b} \: e^{\lambda_a} \Gamma_{ab} \bra{\lambda_a}
D \ket{\lambda_b}
= \sum_{ab} \ketbra{\lambda_a}{\lambda_b} e^{\lambda_a} \Gamma_{ab} (\Lambda^\dagger
D \Lambda)_{ab}\\
&= \sum_{ab} \ketbra{\lambda_a}{\lambda_b} e^{\lambda_a} \bra{a} \left(\Gamma \odot
(\Lambda^\dagger D \Lambda) \right) \ket{b}\\
&= \sum_{ab} e^A \ketbra{\lambda_a}{a} \left(\Gamma \odot
(\Lambda^\dagger D \Lambda) \right) \ketbra{b}{\lambda_b}\\
&= e^A \Lambda \left(\Gamma \odot
(\Lambda^\dagger D \Lambda) \right) \Lambda^\dagger.
\end{align}
$\odot$ and $\oslash$ denote an elementwise (Hadamard) product and division.
With $A = \log(U)$, Eq.~\eqref{eq:invJ} now gives
\be
\VEC^{-1} \left( J(\log)|_U \: \VEC(X) \right)
= 
\Lambda \left((\Lambda^\dagger U^\dagger X \Lambda)
\oslash \Gamma\right) \Lambda^\dagger,
\ee
% check inverse; works both ways
and finally
\begin{align}
\notag
\pde{}{\alpha_{rk}} \mf{L}(f)
&= J(\mf{L}(f))|_f \: \vec{e}_{rk}
= J(P)|_{\log(V^\dagger U_f)} J(\log)|_{V^\dagger U_f} J(V^\dagger U_f)|_{f} \: \vec{e}_{rk}\\
\notag
&= P \: J(\log)|_{V^\dagger U_f}
\VEC \left. \left( V^\dagger \pde{}{\alpha_{rk}} U_f \right)\right|_{f}\\
\notag
&= P \VEC \left( 
\Lambda \left( \left(\Lambda^\dagger U_f^\dagger V
\left. \left( V^\dagger \pde{}{\alpha_{rk}} U_f \right)\right|_{f}
\Lambda \right) \oslash \Gamma \right) \Lambda^\dagger
\right)\\
&= P \VEC \left( 
\Lambda \left( \left(\Lambda^\dagger
\left. \left( V^\dagger \pde{}{\alpha_{rk}} U_f \right)\right|_{f}
\Lambda \right) \oslash \zeta \right) \Lambda^\dagger
\right)
\label{eq:gradientL}
\end{align}
where $\vec{e}_{rk}$ is an unit vector, and $\Lambda$ and $\Gamma$ are obtained from the eigendecomposition
of~$\log(V^\dagger U_f)$ as explained above.

We also have
\be
\left. \left( V^\dagger \pde{}{\alpha_{rk}} U_f \right)\right|_{f}
=
V^\dagger P_N P_{N-1} \cdots P_{k+1} \left. \pde{P_k}{\alpha_{rk}}
\right|_{f} P_{k-1} \cdots P_1,
\ee
and
\be
\left. \pde{P_k}{\alpha_{rk}} \right|_{f}
=
\sum_{ab} \ketbra{\hat{\lambda}_a}{\hat{\lambda}_b}  e^{\hat{\lambda}_a} \hat{\Gamma}_{ab} \bra{\hat{\lambda}_a} (-\tau_k B_r) \ket{\hat{\lambda}_b}
=
-\tau_k P_k \hat{\Lambda} \left(\hat{\Gamma} \odot
(\hat{\Lambda}^\dagger B_r \hat{\Lambda}) \right) \hat{\Lambda}^\dagger.
\ee
NOTE that the $\hat{\Lambda}$ and $\hat{\Gamma}$ matrices here are different from
the ones in Eq.~\eqref{eq:gradientL}, and obtained from the
eigendecomposition of~$-\tau_k H_k$.


Computing the Jacobian:
\begin{align*}
\pde{}{\alpha_{rk}} \mf{L}(f)
&= P \VEC \left( 
\Lambda \left( \left(\Lambda^\dagger
U_{k-1}^\dagger \left[
-\tau_k \hat{\Lambda} \left(\hat{\Gamma} \odot
(\hat{\Lambda}^\dagger B_r \hat{\Lambda}) \right) \hat{\Lambda}^\dagger
\right] U_{k-1}
\Lambda \right) \oslash \Gamma \right) \Lambda^\dagger
\right)\\
&= P \VEC \left( 
\Lambda \left( \left(\Lambda^\dagger
Q_{k}
\left[
-\tau_k \hat{\Lambda} \left(\hat{\zeta} \odot
(\hat{\Lambda}^\dagger B_r \hat{\Lambda}) \right)
\hat{\Lambda}^\dagger
\right]
U_{k-1}
\Lambda \right) \oslash \zeta \right) \Lambda^\dagger
\right).
\end{align*}
%\begin{acknowledgments}
%\end{acknowledgments}

\bibliography{qc}
\end{document}

 
